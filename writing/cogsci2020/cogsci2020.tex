% Template for Cogsci submission with R Markdown

% Stuff changed from original Markdown PLOS Template
\documentclass[10pt, letterpaper]{article}

\usepackage{cogsci}
\usepackage{pslatex}
\usepackage{float}
\usepackage{caption}

% amsmath package, useful for mathematical formulas
\usepackage{amsmath}

% amssymb package, useful for mathematical symbols
\usepackage{amssymb}

% hyperref package, useful for hyperlinks
\usepackage{hyperref}

% graphicx package, useful for including eps and pdf graphics
% include graphics with the command \includegraphics
\usepackage{graphicx}

% Sweave(-like)
\usepackage{fancyvrb}
\DefineVerbatimEnvironment{Sinput}{Verbatim}{fontshape=sl}
\DefineVerbatimEnvironment{Soutput}{Verbatim}{}
\DefineVerbatimEnvironment{Scode}{Verbatim}{fontshape=sl}
\newenvironment{Schunk}{}{}
\DefineVerbatimEnvironment{Code}{Verbatim}{}
\DefineVerbatimEnvironment{CodeInput}{Verbatim}{fontshape=sl}
\DefineVerbatimEnvironment{CodeOutput}{Verbatim}{}
\newenvironment{CodeChunk}{}{}

% cite package, to clean up citations in the main text. Do not remove.
\usepackage{apacite}

% KM added 1/4/18 to allow control of blind submission


\usepackage{color}

% Use doublespacing - comment out for single spacing
%\usepackage{setspace}
%\doublespacing


% % Text layout
% \topmargin 0.0cm
% \oddsidemargin 0.5cm
% \evensidemargin 0.5cm
% \textwidth 16cm
% \textheight 21cm

\title{Child language input does not reflect word frequency: Typical and
atypical feature description across development}


\author{{\large \bf Morton Ann Gernsbacher (MAG@Macc.Wisc.Edu)} \\ Department of Psychology, 1202 W. Johnson Street \\ Madison, WI 53706 USA \AND {\large \bf Sharon J.~Derry (SDJ@Macc.Wisc.Edu)} \\ Department of Educational Psychology, 1025 W. Johnson Street \\ Madison, WI 53706 USA}

\begin{document}

\maketitle

\begin{abstract}
How do children learn the typical features of objects in the world? For
many objects, this information must come from the language they hear.
However, language does not veridically reflect the world: People are
more likely to talk about atypical features (e.g. ``purple carrot'')
than typical features (``e.g.~orange carrot''). Does the speech that
children hear from their parents also overrepresent atypical features?
We estimated the typicality of features described by adjectives produced
by parents in a large, longitudinal corpus of parent-child interaction.
Across nearly 2000 unique adjective--noun pairs, we found that parents
generally highlight atypical features of objects, but also that they are
more likely to describe the typical features of objects when their
children are younger. We also found that vector space models trained on
linguistic input recovered very little of this typicality information.
These results suggest that young children may either have warped
estimates of typical features for categories they learn about through
language or use other information to acquire adult-like typicality
information.

\textbf{Keywords:}
language input, language acquisition, child-directed speech
\end{abstract}

Children learn a tremendous amount about the structure of the world
around them in just a few short years, from the rules that govern the
movement of physical objects to the hierarchical structure of natural
categories and even relational structures among social and cultural
groups (Baillargeon, 1994; Legare \& Harris, 2016; Rogers \& McClelland,
2004). Where does the information driving this rapid acquisition come
from? Undoubtedly, a sizeable component comes from direct experience
observing and interacting with the world (Sloutsky \& Fisher, 2004;
Stahl \& Feigenson, 2015). But another important source of information
comes from the language people use to talk about the world (Landauer \&
Dumais, 1997; Rhodes, Leslie, \& Tworek, 2012). How similar is the
information available from children's direct experience to the
information available in the language children hear?

Two lines of work suggest that they may be surprisingly similar. One
compelling area of work is the comparison of semantic structures learned
by congentinally blind children to those of their sighted peers. In
several domains that would at first blush rely heavily on visual
information, such as color terms and verbs of visual perception (e.g.
\emph{look}, \emph{see}), blind children's semantic similarity judgments
are quite similar to those of sighted children (Landau, Gleitman, \&
Landau, 2009). Further, blind adults' judgments of visual perception
verbs are sensitive to highly detailed information like variation in
intensity (e.g.~blaze vs.~glow), just like sighted adults (Bedny,
Koster-Hale, Elli, Yazzolino, \& Saxe, 2019). A second line of evidence
supporting the similarity of information in perception and language is
the broad success of statistical models trained on language alone in
approximating human judgments across a variety of domains (Landauer \&
Dumais, 1997; Mikolov, Sutskever, Chen, Corrado, \& Dean, 2013). Even
more compellingly, models trained on both linguistic usage and
perceptual features for some words can infer the perceptual features of
linguistically related words entirely from the covariation of language
and perception (Johns \& Jones, 2012).

Still, there is reason to believe that some semantic features may be
harder to learn from language than these data suggest. This is because
people rarely use language merely to provide running comentary on the
world around them; instead, we use language to talk about things that
diverge from our expectations or those of our conversational partner
(Grice, 1975). People tend to avoid being over- or under-informative
when they speak, for instance noting only the features of an object
necessary for their conversational partner to identify it (Mangold \&
Pobel, 1988). In particular, people are informative with respect to both
the referential context and the typical features of the the object to
which they are referring (Rubio-Fernández, 2016; Westerbeek, Koolen, \&
Maes, 2015). People tend to refer to an object that is typical of its
category with a bare noun (e.g., calling an orange carrot ``a carrot''),
but often specify when an object has an atypical feature (e.g, ``a
purple carrot''). Given these communicative pressures, naturalistic
language statistics may provide surprisingly little evidence about what
is typical (Willits, Sussman, \& Amato, 2008).

If parents speak to children in this minimally informative way, children
may be faced with input that emphasizes atypicality in relation to world
knowledge they do not yet have. For things like carrots--which children
learn about both from perception and from language--this issue may be
resolved by integrating both sources of information. Likely almost all
of the carrots children see are orange, and hearing an atypical exemplar
noted as a ``purple carrot'' may make little difference in their
inferences about the category of carrots more broadly. But for things to
which they lack perceptual access--such as rare objects, unfamiliar
social groups, or inaccessible features like the roundness of the
Earth--much of the information must come from language (Harris \&
Koenig, 2006). If language predominantly notes atypical features rather
than typical ones, children may overrepresent atypical features as they
learn the way things in the world tend to be.

On the other hand, parents may speak to children far differently from
the way they speak to other adults. Parents' speech may reflect typical
features of the world more veridically, or even emphasize typical
features in order to teach children about the world. Parents alter their
speech to children along a number of structural dimensions, using
simpler syntax and more reduplications (Snow, 1972). Their use of
description may reflect similar alignment to children's growing
knowledge.

We examine the typicality of adjectives in a large, diverse corpus of
parent-child interactions recorded in children's homes to ask whether
parents talking to their children--just like adults speaking and
writing--tend to use adjectives predominantly to mark atypical features.
We find that they do: Parents and children overwhelmingly choose to
mention atypical rather than typical features. We also find that parents
use adjectives differently over the course of children's development,
noting typical features more often to younger children. We then ask
whether the co-occurrence structure of language nonetheless captures
typicality information by training vector space models on child-directed
speech. We find that relatively little typical feature information is
represented in these semantic spaces.

\hypertarget{adjective-typicality}{%
\section{Adjective typicality}\label{adjective-typicality}}

In order to determine whether parents use adjectives mostly to mark
atypical features of categories, we analyzed caregiver speech from a
large corpus of parent-child interactions recorded in the home. We
extracted a subset of adjective-noun combinations that co-occured, and
asked a sample of Amazon Mechanical Turkers to judge how typical the
property described by each adjective was for the noun it modified. We
then examined both the broad features of this typicality distribution
and how it changed over development.

\hypertarget{corpus}{%
\subsection{Corpus}\label{corpus}}

We used data from the Language Development Project-- a large-scale,
longitudinal corpus of parent-child interactions recorded in children's
homes. Families were recruited to be representative of the Chicagoland
area in both socio-economic and racial composition (Goldin-Meadow et
al., 2014). Recordings were taken in the home every 4-months from when
the child was 14-months-old until they were 58-months-old, resulting in
12 timepoints. Each recording was of a 90-minute session in which
parents and children were free to behave as they liked and interact as
much or as little as they liked.

Our sample consisted of 64 typically-developing children and their
caregivers, with data from at least 4 timepoints (\emph{mean} = 11.3
timepoints). Together, this resulted in a total of 641,402 distinct
parent utterances.

\hypertarget{stimulus-selection}{%
\subsection{Stimulus Selection}\label{stimulus-selection}}

From these utterances, we extracted all of the nouns (using human-coded
part of speech tags) resulting in a set of 8,150 total nouns. Because of
our interest in change over development, we considered only nouns that
appeared at least once every 3 sessions (i.e.~per developmental year).
This yielded a set of some 1,829 potential target nouns used over
198,014 distinct utterances.

We selected from the corpus all 35,761 distinct utterances containing
any of these nouns and any word tagged as an adjective. We considered
for analysis all adjective-noun pairs that occurred in any utterance
(i.e.~utterances with one noun and three adjectives were coded as three
pairs) for a total of 18,050 distinct pairs. This set contained a number
of high-frequency idiomatic pairs whose typicality was difficult to
classify (e.g., ``good''--``job''; ``little''--``bit''). To resolve this
issue, we used human judgments of words' concreteness to identify and
exclude candidate idioms (Brysbaert, Warriner, \& Kuperman, 2014). We
retained for analysis only pairs where both the adjective and noun were
in the top 25\% of the concreteness ratings (e.g., ``dirty'' --
``dish''; ``green'' -- ``fish'') restricting our set to 2,477. Finally,
all pairs were given to 7 human coders to judge whether the pair was
``incoherent or unrelated'' and we thus excluded a final 576 pairs from
the sample (e.g., incoherent pairs such as ``flat'' -- ``honey'').

Thus, our final sample included 1,901 unique adjective-noun pairs drawn
from 3,749 distinct utterances. The pairs were combinations of 637
distinct concrete nouns and 111 distinct concrete adjectives. We
compiled these pairs and collected human judgments on Amazon Mechanical
Turk for each pair, as described below. Table \ref{tab:utt_table}
contains example utterances from the final set and typicality judgments
from our human raters.

\begin{table*}[tb]
\centering
\begin{tabular}{llrrrrr}
  \hline
utterance & pair & rating 1 & rating 2 & rating 3 & rating 4 & mean typicality \\ 
  \hline
especially with wooden shoes. & wooden-shoe &   2 &   2 &   3 &   2 & 2.75 \\ 
  you like red onions? & red-onion &   3 &   5 &   3 &   4 & 3.75 \\ 
  the garbage is dirty. & dirty-garbage &   7 &   7 &   5 &   7 & 6.50 \\ 
   \hline
\end{tabular}
\caption{Sample typicality ratings from 4 human coders for three adjective-noun pairs drawn from the corpus.} 
\label{tab:utt_table}
\end{table*}

\hypertarget{participants}{%
\subsection{Participants}\label{participants}}

Each participant rated 20 pairs, and each pair was rated by four
participants; we used
\href{http://docs.dallinger.io/en/latest/}{Dallinger}, a tool for
automating complex recruitment on Amazon Mechanical Turk, to balance
recruitment. Overall, we recruited 444 participants to rate 2200
adjective--noun pairs. After exclusions using an attention check
\textbf{what was the check?}, we retained 8580 judgments, with each
adjective--noun pair retaining at least two judgments.

\hypertarget{design-and-procedure}{%
\subsection{Design and Procedure}\label{design-and-procedure}}

To evaluate the typicality of the adjective--noun pairs that appeared in
parents' speech, we asked participants on Amazon Mechanical Turk to rate
each pair. Participants were presented with a question of the form ``How
common is it for a cow to be a brown cow?'' and asked to provide a
rating on a seven-point scale: (1) never, (2) rarely, (3) sometimes, (4)
about half the time, (5) often, (6) almost always, (7) always. These
ratings were combined with usage data from our corpus analysis to let us
determine the extent to which parents use language to describe typical
and atypical features.

\begin{CodeChunk}
\begin{figure}[tb]

{\centering \includegraphics{figs/distribution_plot-1} 

}

\caption[Denisity plots showing the usage amount at each timepoint based on the typicality of the adj-noun pair]{Denisity plots showing the usage amount at each timepoint based on the typicality of the adj-noun pair.}\label{fig:distribution_plot}
\end{figure}
\end{CodeChunk}

\hypertarget{results}{%
\subsection{Results}\label{results}}

To estimate the relative frequency with which caregivers refer to
typical vs.~atypical referents, we used Turker's judgments about the
atypicality of each adjective-noun pair in the data. In our analyses, we
token-weighted these judgments--giving higher weight to pairs that
occurred more frequently in children's inputs. However, results are
qualitatively identical and all signiifcant effects remain significant
without these re-weightings.

If caregivers speak informatively to convey what is atypical or
surprising in relation to their own sophisticated world knowledge, we
should see that caregiver description is dominated by modifiers that are
sometimes true of the noun they modify. If instead child-directed speech
privledges redundant information perhaps to align to young children's
limited world knowledge, caregiver description should yield a distinct
distribution dominated by highly typical modifiers. As can be seen in
Figure \ref{fig:distribution_plot} parents' description largely focuses
on features that are only sometimes true of the concept.

To confirm this effect statistically, we first centered the ratings
(i.e. ``about half'' was coded 0), and then predicted the rating on each
trial with a mixed effect model with only an intercept and a random
effect of noun (\texttt{typicality $\sim$ 1 + (1|noun)}). The intercept
was reliably negative, indicating that adjective tends to refer to
atypical features of objects (\(\beta =\) -0.77, \(t =\) -19.72, \(p\)
\textless{} .001). We then re-estimated these models seperately for each
age in the corpus, and found a reliablly negative intercept for every
age group (smallest effect \(\beta_{14} =\) -0.5, \(t =\) -4.45, \(p =\)
\textless{} .001). These data suggest that even when talking with very
young children, caregiver speech is structured according to
communicative pressures observed in the lab.

For comparison, we performed the same analyses but with typicality
judgments weighted not by the frquency of each adjective-noun pair's
occurrence in the Language Development Project, but instead by their
frequency of occurrence in the Corpus of Contemporary American English
(COCA; Davies, 2008). While this estimate of adult usage is
imperfect--as the adjective-nouns pairs produced by parents in our
corpus may not be a representative sample of adjectives and nouns spoken
by the adults in COCA, it provides a first approximation to adult usage.
When we fit the same mixed-effects model to the data, we found that the
intercept was reliably negative, indicating that adult-directed peech is
likely also biased toward description of atypical features (\(\beta =\)
-0.3, \(t =\) -19.72, \(p\) \textless{} .001)

Returning to caregiver speech, while descriptions at every age tended to
point out atypical features as in adult-direct speech, this effect
changed in strength over development. An age effect added to the
previous model was reliably negative, indicating that parents of older
children are relatively more likely to focus on atypical features
(\(\beta =\) -0.11, \(t =\) -3.47, \(p =\) .001). In line with our
hypotheses, it seems that caregivers are more likely to provide
description of typical features for their young children, compared with
older children. As a second test of this intuition, we defined
adjectives as highly-typical if Turkers judged them to be `often',
`almost always', or `always' true. We predicted whether each judgment
was highly typical from a mixed-effects logistic regression with a fixed
effect of age (log-scaled) and a random effect of noun. Age was a
highly-reliable predictor (\(\beta =\) -0.94, \(t =\) -5.01, \(p =\)
\textless{} .001). While children at all ages hear more talk about what
is atypically true (Figure \ref{fig:distribution_plot}), younger
children hear relatively more talk about what is typically true than
older children do (Figure \ref{fig:prototypical_plot}).

\begin{CodeChunk}
\begin{figure}[tb]

{\centering \includegraphics{figs/prototypical_plot-1} 

}

\caption[Proportion of caregiver description that is about typically-true features, as a function of age]{Proportion of caregiver description that is about typically-true features, as a function of age.}\label{fig:prototypical_plot}
\end{figure}
\end{CodeChunk}

\hypertarget{child-speech.}{%
\subsubsection{Child Speech.}\label{child-speech.}}

Given the striking consistency in adult-to-adult speech and caregiver
speech across ages, we next briefly consider what kind of information is
contained in children's speech. By analyzing children's own utterances,
we can determine when children come to use description in a way that
looks like caregiver speech. Are children mirroring adult-like uses of
description even from a young age, or are they choosing to describe more
typical features of the world?

The Language Development Corpus contains 442,048 child utterances. Using
the set of adjective-noun pairs for which we have judgments from our
analysis of caregiver speech, we repeat our analysis on usage data for a
set of 543 distinct adjective-noun pairs, 494 of which appeared in
children's productions. While preliminary, a mixed effects model
predicting typicality had a highly-reliable negative intercept
(\(\beta =\) -0.71, \(t =\) -10.3, \(p =\) \textless{} .001), but adding
an age term did not improve model fit. Thus, children's speech is also
biased towards atypical descriptions, and this bias does not change
reliably over the first 5 years.

\hypertarget{discussion}{%
\subsection{Discussion}\label{discussion}}

In sum, we find robust evidence that language is used to discuss
atypical, rather than typical, features of the world. Description in
caregiver speech seems to largely mirror the usage patterns that we
observed in adult-adult speech, suggesting that these patterns arise
from general communicative pressures. Indeed, even children's own
productions show a similar usage pattern, with more description of
atypical features of the world as early as we can measure.

It should be noted that children's utterances come from naturalistic
conversations with caregivers, and their use of atypical description may
be prompted by parent-led discourse. That is, if a caregiver chooses to
describe the \emph{purpleness} of a cat in book, the child may well
respond by asking about that same feature. Future analyses would need to
better disentangle the extent to which children's productions are
imitative of caregivers.

Interestingly, the descriptions children hear change over development,
becoming increasingly focused on atypical features. The higher
prevalance of typical descriptors in early development may help young
learners learn what is typical; however, even at the earliest point we
measured, the bulk of language input describes atypical features.

This usage pattern aligns with the idea that language is used
informatively in relation to background knowledge about the world. It
may pose a problem, however, for young language learners with
still-developing world knowledge. If language does not transparently
convey the typical features of objects, and instead (perhaps
misleadingly) notes the atypical ones, how might children come to learn
what objects are typically like? One possibility is that information
about typical features is captured in regularities across many
utterances. If this is true, language may still be an important source
of information about typicality as children may be able to extract more
accurate typicality information by tracking second-order co-occurence.

\hypertarget{extracting-structure-from-language}{%
\section{Extracting Structure from
Language}\label{extracting-structure-from-language}}

Much information can be gleaned from language that does not seem
available at first glance. From language alone, simple distributional
learning models can recover enough information to perform comparably to
non-native college applicants on the Test of English as a Foreign
Language (Landauer \& Dumais, 1997). Recently, Lewis, Zettersten, \&
Lupyan (2019) demonstrated that even nuanced feature information may be
learnable through distributional semantics alone, without any complex
inferential machinery. We take a similar approach to ask whether a
distributional semantics model trained on the language children hear can
capture typical feature information.

\begin{CodeChunk}
\begin{figure}[tb]

{\centering \includegraphics{figs/halfs-1} 

}

\caption[Plots of nouns for which there was at least one atypical adjective (rated at most "sometimes"), and at least one typical adjective ("rated at least often")]{Plots of nouns for which there was at least one atypical adjective (rated at most "sometimes"), and at least one typical adjective ("rated at least often"). The black dotted line shows average human typicality ratings (scaled) for these items. The blue line shows how well our models do at capturing this trend, with grey lines representing individual pairs.}\label{fig:halfs}
\end{figure}
\end{CodeChunk}

\begin{table}[tb]
\centering
\begin{tabular}{lll}
  \hline
noun & high & low \\ 
  \hline
apple & red & orange \\ 
  lemon & yellow & orange \\ 
  puzzle & flat & giant \\ 
  bird & outside & purple \\ 
  banana & yellow & orange \\ 
  elephant & fat & pink \\ 
  whale & wet & red \\ 
  frog & green & purple \\ 
  food & solid & purple \\ 
  balloon & colored & orange \\ 
   \hline
\end{tabular}
\caption{The ten cases in which word2vec similarities are worst at predicting human typicality judgments. For each of these cases, the model judges the low-typicality adjective to be more similar to the noun than the high-typicality adjective.} 
\label{tab:pairs_tab}
\end{table}

\hypertarget{method}{%
\subsection{Method}\label{method}}

To test this possibility, we trained word2vec, a model that predicts
words using their contexts, on the same corpus of child-directed speech
used in our first set of analyses. Our model is a
continuous-bag-of-words word2vec model trained using the package gensim
(Řehůřek \& Sojka, 2010). If the model captures information about the
typical features of objects, we should see that the model's word pair
similarities are correlated with the typicality ratings we elicited from
human raters. For a second comparison, we also used an off-the-shelf
implementation of word2vec trained on Wikipedia (Mikolov, Grave,
Bojanowski, Puhrsch, \& Joulin, 2018). While the Language Development
Project corpus likely underestimates the amount of structure in
children's linguistic input, Wikipedia likely overestimates it.

\hypertarget{results-1}{%
\subsection{Results}\label{results-1}}

We find that similarities in the model trained on the Language
Development Project corpus have near zero correlation with human
adjective--noun typicality ratings (\(r =\) 0, \(p =\) .865). This is in
spite of better correlations with large sets of human similarity
judgments between different kinds of word pairs (correlation with
wordsim353, 0.37; correlation with simlex, 0.15). This suggests that
statistical patterns in child-directed speech are likely insufficient to
encode information about the typical features of objects, despite
encoding at least some information about word meaning more broadly.

However, the corpus on which we trained this model was small; perhaps
our model did not get enough language to draw out the patterns that
would reflect the typical features of objects. To test this possibility,
we asked whether word vectors trained on a much larger corpus---English
Wikipedia---strongly correlate with typicality ratings. This model's
similarities were signifcantly correlated with human judgments, although
the strength of the correlation was still fairly weak (\(r =\) 0.23,
\(p\) \textless{} .001). Interestingly, similarities from the two models
wrere more highly correlated than either model's similarity and human
judgments (\(r =\) 0.31, \(p\) \textless{} .001). This suggests that
these models are picking up on some systematic associations between
nouns and adjectives, but not typicality inofrmation ones.

One possibile confound in these analyses is that the similarity
judgments produced by our models reflect many dimensions of similarity,
but human judgments reflect only typicality. To accomodate this, we
performed a second analysis where we considered only the subset of 79
nouns that had both a typical (rated as at least ``often'') and an
atypical (rated as at most ``sometimes'') adjective. We then asked
whether the models rated the typical adjective as more similar to the
noun it modified than the atypical noun. The LDP model correctly
classified 42 out of 79 (0.53), which was not better than chance
(\(p =\) .653). The Wikipedia model correctly classified 58 out of 79
(0.73), which was better than chance according to a binomial test, but
still fairly poor perfomance (\(p =\) \textless{} .001). Fig
\ref{fig:halfs} shows the ratings from Turkers and the two models for
the 79 nouns. Table \ref{tab:pairs_tab} shows the 10 adjective-noun
pairs where word2vec trained on Wikipedia produce judgments that are
farthest away from the human judgments for these nouns (LDP was
similar). \textbf{fix up the discussion here}

\hypertarget{general-discussion}{%
\section{General Discussion}\label{general-discussion}}

Language provides children a rich source of information about the world.
However, this information is not always transparently available: becuase
language is used to comment on the atypical and surprising, it does not
perfectly mirror the world. Among adult conversational partners whose
world knowledge is well-aligned, this characteristic of language allows
people to converse informatively and not redundantly. But between a
child and caregiver whose world knowledge is asymmetric, this pressure
competes with other demands: what is minimally informative to an adult
may be misleading to a child. Our results show that this pressure
structures language to create a peculiar learning environment; one in
which caregivers predominantly point out the atypical features of
things.

How, then, do children learn about the typical features of things from
such an environment? While younger children may gain an important
foothold from hearing more description of typical features, they still
face language dominated by atypical description. When we looked at more
nuanced ways of extracting information from language (which may or may
not be available to the developing learner), we found that models of
distributional semantics capture little typical feature information.

Of course, one source of information that may simplify this problem is
perceptual information from the world itself. In many cases, perceptual
information may swamp information from language; children likely see
enough orange carrots in the world to outweigh hearing ``purple
carrot.'' It remains unclear, however, how children learn about
categories for which they have scarcer evidence. Indeed, language
information likely swamps perceptual information for many other
categories, such as abstract concepts or those that cannot be learned
about by direct experience. Given that the present work is limited to
concrete concepts, we can only speculate about the information
caregivers provide about abstract concepts. But, if abstract concepts
pattern similarly to concrete objects, children are in a particularly
difficult bind. Though perceptual information is undoubtedly useful in
learning about the typical features of things, it remains to be
explained how children learn what is typical when this perceptual
information is scant, irrelevant, or incomplete.

Another possibility is that children expect language to be used
informatively at a young age. Under this hypothesis, their language
environment is not misleading at all. If young children expect
adjectives to mark atypical features, they can use description and the
lack thereof to learn more about the world around them. Lab studies find
that children expect interlocutors to be informative in relation to
their prior knowledge by the age of 2 (Akhtar, Carpenter, \& Tomasello,
1996) and that 5--6-year-old children are roughly informative with
respect to their interlocutor's perspective in a referential
communication task (Nadig \& Sedivy, 2002). The idea that young children
understand the informative purpose of language is consistent with our
finding that even young children also largely choose to describe
atypical features. Though this effect can be explained by simpler means
such as salience or mimicry, it suggests that caregivers and children
may be usefully aligned in the aspects of the world they choose to talk
about.

Whether adult-directed, child-directed, or a child's own speech,
language is used with remarkable consistency: people talk about the
atypical. Though parents might reasonably be broadly over-informative in
order to teach their children about the world, this is not the case.
This presents a potential puzzle for young learners who have limited
world knowledge and limited pragmatic inferential abilities. Perceptual
information and nascent pragmatic abilities may help fill in the gaps,
but much remains to be explored to link these explanations to actual
learning. Communication pressures are pervasive forces structuring the
language children hear, and future work must disentangle whether
children capitalize on them or are misled by them in learning about the
world.

\vspace{1em} \fbox{\parbox[b][][c]{7.3cm}{\centering Data and analysis code will be made available through GitHub after de-anonymization. \ }}

\hypertarget{references}{%
\section{References}\label{references}}

\setlength{\parindent}{-0.1in} 
\setlength{\leftskip}{0.125in}

\noindent

\hypertarget{refs}{}
\leavevmode\hypertarget{ref-akhtar1996}{}%
Akhtar, N., Carpenter, M., \& Tomasello, M. (1996). The Role of
Discourse Novelty in Early Word Learning. \emph{Child Development},
\emph{67}(2), 635--645.

\leavevmode\hypertarget{ref-baillargeon1994}{}%
Baillargeon, R. (1994). How do infants learn about the physical world?
\emph{Current Directions in Psychological Science}, \emph{3}(5),
133--140.

\leavevmode\hypertarget{ref-bedny2019}{}%
Bedny, M., Koster-Hale, J., Elli, G., Yazzolino, L., \& Saxe, R. (2019).
There's more to ``sparkle'' than meets the eye: Knowledge of vision and
light verbs among congenitally blind and sighted individuals.
\emph{Cognition}, \emph{189}, 105--115.

\leavevmode\hypertarget{ref-brysbaert2014}{}%
Brysbaert, M., Warriner, A. B., \& Kuperman, V. (2014). Concreteness
ratings for 40 thousand generally known english word lemmas.
\emph{Behavior Research Methods}, \emph{46}(3), 904--911.

\leavevmode\hypertarget{ref-davies2008}{}%
Davies, M. (2008). The corpus of contemporary american english (coca):
520 million words, 1990-present.

\leavevmode\hypertarget{ref-goldin-meadow2014}{}%
Goldin-Meadow, S., Levine, S. C., Hedges, L. V., Huttenlocher, J.,
Raudenbush, S. W., \& Small, S. L. (2014). New evidence about language
and cognitive development based on a longitudinal study: Hypotheses for
intervention. \emph{American Psychologist}, \emph{69}(6), 588.

\leavevmode\hypertarget{ref-grice1975}{}%
Grice, H. P. (1975). Logic and conversation. In \emph{Speech acts} (pp.
41--58). Brill.

\leavevmode\hypertarget{ref-harris2006}{}%
Harris, P. L., \& Koenig, M. A. (2006). Trust in testimony: How children
learn about science and religion. \emph{Child Development},
\emph{77}(3), 505--524.

\leavevmode\hypertarget{ref-johns2012}{}%
Johns, B. T., \& Jones, M. N. (2012). Perceptual inference through
global lexical similarity. \emph{Topics in Cognitive Science},
\emph{4}(1), 103--120.

\leavevmode\hypertarget{ref-landau2009}{}%
Landau, B., Gleitman, L. R., \& Landau, B. (2009). \emph{Language and
experience: Evidence from the blind child} (Vol. 8). Harvard University
Press.

\leavevmode\hypertarget{ref-landauer1997}{}%
Landauer, T. K., \& Dumais, S. T. (1997). A solution to plato's problem:
The latent semantic analysis theory of acquisition, induction, and
representation of knowledge. \emph{Psychological Review}, \emph{104}(2),
211.

\leavevmode\hypertarget{ref-legare2016}{}%
Legare, C. H., \& Harris, P. L. (2016). The ontogeny of cultural
learning. \emph{Child Development}, \emph{87}(3), 633--642.

\leavevmode\hypertarget{ref-lewis2019}{}%
Lewis, M., Zettersten, M., \& Lupyan, G. (2019). Distributional
semantics as a source of visual knowledge. \emph{Proceedings of the
National Academy of Sciences}, \emph{116}(39), 19237--19238.

\leavevmode\hypertarget{ref-mangold1988}{}%
Mangold, R., \& Pobel, R. (1988). Informativeness and Instrumentality in
Referential Communication. \emph{Journal of Language and Social
Psychology}, \emph{7}(3-4), 181--191.

\leavevmode\hypertarget{ref-mikolov2018}{}%
Mikolov, T., Grave, E., Bojanowski, P., Puhrsch, C., \& Joulin, A.
(2018). Advances in pre-training distributed word representations. In
\emph{Proceedings of the international conference on language resources
and evaluation (lrec 2018)}.

\leavevmode\hypertarget{ref-mikolov2013}{}%
Mikolov, T., Sutskever, I., Chen, K., Corrado, G. S., \& Dean, J.
(2013). Distributed representations of words and phrases and their
compositionality. In \emph{Advances in neural information processing
systems} (pp. 3111--3119).

\leavevmode\hypertarget{ref-nadig2002}{}%
Nadig, A. S., \& Sedivy, J. C. (2002). Evidence of Perspective-Taking
Constraints in Children's On-Line Reference Resolution.
\emph{Psychological Science}, \emph{13}(4), 329--336.

\leavevmode\hypertarget{ref-rhodes2012}{}%
Rhodes, M., Leslie, S.-J., \& Tworek, C. M. (2012). Cultural
transmission of social essentialism. \emph{Proceedings of the National
Academy of Sciences}, \emph{109}(34), 13526--13531.

\leavevmode\hypertarget{ref-rogers2004}{}%
Rogers, T. T., \& McClelland, J. L. (2004). \emph{Semantic cognition: A
parallel distributed processing approach}. MIT press.

\leavevmode\hypertarget{ref-rubio-fernandez2016}{}%
Rubio-Fernández, P. (2016). How Redundant Are Redundant Color
Adjectives? An Efficiency-Based Analysis of Color Overspecification.
\emph{Frontiers in Psychology}, \emph{7}.

\leavevmode\hypertarget{ref-rehurek2010}{}%
Řehůřek, R., \& Sojka, P. (2010). Software Framework for Topic Modelling
with Large Corpora. In \emph{Proceedings of the LREC 2010 Workshop on
New Challenges for NLP Frameworks} (pp. 45--50). Valletta, Malta: ELRA.

\leavevmode\hypertarget{ref-sloutsky2004}{}%
Sloutsky, V. M., \& Fisher, A. V. (2004). Induction and categorization
in young children: A similarity-based model. \emph{Journal of
Experimental Psychology: General}, \emph{133}(2), 166.

\leavevmode\hypertarget{ref-snow1972}{}%
Snow, C. E. (1972). Mothers' speech to children learning language.
\emph{Child Development}, 549--565.

\leavevmode\hypertarget{ref-stahl2015}{}%
Stahl, A. E., \& Feigenson, L. (2015). Observing the unexpected enhances
infants' learning and exploration. \emph{Science}, \emph{348}(6230),
91--94.

\leavevmode\hypertarget{ref-westerbeek2015}{}%
Westerbeek, H., Koolen, R., \& Maes, A. (2015). Stored object knowledge
and the production of referring expressions: The case of color
typicality. \emph{Frontiers in Psychology}, \emph{6}.

\leavevmode\hypertarget{ref-willits2008}{}%
Willits, J. A., Sussman, R. S., \& Amato, M. S. (2008). Event knowledge
vs. Verb knowledge. In \emph{Proceedings of the 30th annual conference
of the cognitive science society} (pp. 2227--2232).

\bibliographystyle{apacite}


\end{document}
