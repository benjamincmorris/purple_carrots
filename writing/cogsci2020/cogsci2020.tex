% Template for Cogsci submission with R Markdown

% Stuff changed from original Markdown PLOS Template
\documentclass[10pt, letterpaper]{article}

\usepackage{cogsci}
\usepackage{pslatex}
\usepackage{float}
\usepackage{caption}

% amsmath package, useful for mathematical formulas
\usepackage{amsmath}

% amssymb package, useful for mathematical symbols
\usepackage{amssymb}

% hyperref package, useful for hyperlinks
\usepackage{hyperref}

% graphicx package, useful for including eps and pdf graphics
% include graphics with the command \includegraphics
\usepackage{graphicx}

% Sweave(-like)
\usepackage{fancyvrb}
\DefineVerbatimEnvironment{Sinput}{Verbatim}{fontshape=sl}
\DefineVerbatimEnvironment{Soutput}{Verbatim}{}
\DefineVerbatimEnvironment{Scode}{Verbatim}{fontshape=sl}
\newenvironment{Schunk}{}{}
\DefineVerbatimEnvironment{Code}{Verbatim}{}
\DefineVerbatimEnvironment{CodeInput}{Verbatim}{fontshape=sl}
\DefineVerbatimEnvironment{CodeOutput}{Verbatim}{}
\newenvironment{CodeChunk}{}{}

% cite package, to clean up citations in the main text. Do not remove.
\usepackage{apacite}

% KM added 1/4/18 to allow control of blind submission


\usepackage{color}

% Use doublespacing - comment out for single spacing
%\usepackage{setspace}
%\doublespacing


% % Text layout
% \topmargin 0.0cm
% \oddsidemargin 0.5cm
% \evensidemargin 0.5cm
% \textwidth 16cm
% \textheight 21cm

\title{Child language input does not reflect world frequency: Typical and
atypical feature description across development}


\author{{\large \bf Morton Ann Gernsbacher (MAG@Macc.Wisc.Edu)} \\ Department of Psychology, 1202 W. Johnson Street \\ Madison, WI 53706 USA \AND {\large \bf Sharon J.~Derry (SDJ@Macc.Wisc.Edu)} \\ Department of Educational Psychology, 1025 W. Johnson Street \\ Madison, WI 53706 USA}

\begin{document}

\maketitle

\begin{abstract}
Language provides children a powerful source of information about the
world. From language alone, simple distributional learning models can
recover enough information to perform comparably to non-native college
applicants on the TOEFL (Landauer \& Dumais, 1997). Blind children learn
the same kinds of relationships among perceptual categories as sighted
children, without any of the relevant visual input (Landau \& Gleitman,
1985). However, language does not perfectly reflect the world: the most
typical features of natural kinds may often go unremarked. For instance,
adults rarely describe the color of an orange carrot, as world knowledge
makes this description redundant. Given children's nascent world
knowledge, does parents' speech to children follow this pattern? From
longitudinal corpus data of parent-child communication (Goldin-Meadow et
al., 2014) between 14--58 months, we extracted usage data for 684
high-frequency concrete nouns and co-occurring adjectives. Independent
raters coded the typicality of over 2,000 unique adjective--noun pairs
on a 7-point Likert scale (interrater reliability: r = 0.8 in a subset
of the data). If language statistics reflect world statistics,
description should be dominated by the typical (strong negative skew);
however, across all ages, we see descriptors concentrated in the
atypical range (positive skewness = 0.38). Parents were reliably more
likely to use typical descriptors when talking to younger rather than
older children. Overall, child language input reflects notable more than
typical features, but increased description of typical features early in
development may provide a foothold for young learners.

\textbf{Keywords:}
referential pacts; parent-child communication
\end{abstract}

Children learn a tremendous amount about the structure of the world
around in just a few short years. From the rules that govern the
movement of physical objects to the hierarchical structure of natural
categories and even relational structures among social and cultural
groups (Baillargeon, 1994; Legare \& Harris, 2016; Rogers \& McClelland,
2004). Where does the information for this rapid acquisition come from?
Undoubtedly, a sizeable comoponent in at least some domains comes from
direct experience observing and interacting with the world (Sloutsky \&
Fisher, 2004; Stahl \& Feigenson, 2015). But, another important source
of information arises from cultural learning: the information in the
language of people around them (Landauer \& Dumais, 1997; Rhodes,
Leslie, \& Tworek, 2012).

\hypertarget{experiment-1-judgements}{%
\section{Experiment 1: Judgements}\label{experiment-1-judgements}}

\hypertarget{method}{%
\subsection{Method}\label{method}}

\hypertarget{participants}{%
\subsubsection{Participants}\label{participants}}

\hypertarget{stimuli}{%
\subsubsection{Stimuli}\label{stimuli}}

\hypertarget{procedure}{%
\subsubsection{Procedure}\label{procedure}}

\hypertarget{design}{%
\subsubsection{Design}\label{design}}

\hypertarget{results}{%
\subsection{Results}\label{results}}

\hypertarget{acknowledgements}{%
\section{Acknowledgements}\label{acknowledgements}}

Place acknowledgments (including funding information) in a section at
the end of the paper.

\hypertarget{references}{%
\section{References}\label{references}}

\setlength{\parindent}{-0.1in} 
\setlength{\leftskip}{0.125in}

\noindent

\hypertarget{refs}{}
\leavevmode\hypertarget{ref-baillargeon1994}{}%
Baillargeon, R. (1994). How do infants learn about the physical world?
\emph{Current Directions in Psychological Science}, \emph{3}(5),
133--140.

\leavevmode\hypertarget{ref-landauer1997}{}%
Landauer, T. K., \& Dumais, S. T. (1997). A solution to plato's problem:
The latent semantic analysis theory of acquisition, induction, and
representation of knowledge. \emph{Psychological Review}, \emph{104}(2),
211.

\leavevmode\hypertarget{ref-legare2016}{}%
Legare, C. H., \& Harris, P. L. (2016). The ontogeny of cultural
learning. \emph{Child Development}, \emph{87}(3), 633--642.

\leavevmode\hypertarget{ref-rhodes2012}{}%
Rhodes, M., Leslie, S.-J., \& Tworek, C. M. (2012). Cultural
transmission of social essentialism. \emph{Proceedings of the National
Academy of Sciences}, \emph{109}(34), 13526--13531.

\leavevmode\hypertarget{ref-rogers2004}{}%
Rogers, T. T., \& McClelland, J. L. (2004). \emph{Semantic cognition: A
parallel distributed processing approach}. MIT press.

\leavevmode\hypertarget{ref-sloutsky2004}{}%
Sloutsky, V. M., \& Fisher, A. V. (2004). Induction and categorization
in young children: A similarity-based model. \emph{Journal of
Experimental Psychology: General}, \emph{133}(2), 166.

\leavevmode\hypertarget{ref-stahl2015}{}%
Stahl, A. E., \& Feigenson, L. (2015). Observing the unexpected enhances
infants' learning and exploration. \emph{Science}, \emph{348}(6230),
91--94.

\bibliographystyle{apacite}


\end{document}
